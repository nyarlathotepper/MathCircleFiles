\documentclass[12pt]{article}

\usepackage{amssymb}

\usepackage{amsmath}

\usepackage{geometry}

\usepackage{graphicx}

%\usepackage{pst-poly}
%\usetikzlibrary{shapes}

%\usepackage{tikz}

\newcommand{\N}{\mathbb{N}}

\newcommand{\Z}{\mathbb{Z}}

\newcommand{\Q}{\mathbb{Q}}

\newcommand{\R}{\mathbb{R}}

\newcommand{\C}{\mathbb{C}}

\newcommand{\Zp}{\mathbb{Z}_p}

\newcommand{\numb}[1]{\noindent{\bf #1)}}

\newcommand{\n}{\Vert}

\newcommand{\ip}[2]{\langle #1, #2\rangle}

\usepackage{xcolor}

\usepackage[framemethod=tikz]{mdframed}

\definecolor{cccolor}{rgb}{.67,.7,.67}





\begin{document}



\centerline{\bf How to Be Perfect}


\bigskip

%defining perfect numbers and making sure students understand divisibility

\bigskip

\numb{1} We'll start by looking at the whole numbers from 2 to 10.

\bigskip

a) For each number, find all of its \textit{proper} divisors, meaning all divisors that are not equal to the number itself. 

\bigskip

b) For each number, add up all of its proper divisors.

\bigskip

c) How many whole numbers from 2 to 10 are equal to the sum of their proper divisors? Such a number is called \textit{perfect}.


\newpage
%extending to definitions of deficient and abundant numbers.

\numb{2} Recapping, a whole number that is the sum of its proper divisors is called \textit{perfect}. A number that is bigger than the sum of its proper divisors is called \textit{deficient} and a number that is smaller than the sum of its proper divisors is called \textit{abundant}.

\bigskip

a) How many of the whole numbers from 11 to 20 are perfect, deficient, or abundant?

\bigskip

b) How many of the whole numbers from 21 to 30 are perfect, deficient, or abundant?



\newpage 

%getting students to think about how to generate infinitely many examples; the last part is still an open question, so don't let them work too long on it!

\numb{3} Now think of all the whole numbers, of which there are infinitely many.

\bigskip

a) How many deficient numbers do you think there are? Can you convince everyone that your answer is true? 

\bigskip

b) Same question for abundant numbers. 

\bigskip

c) Same question for perfect numbers. 


\newpage

%beginning the connection between perfect numbers and Mersenne primes.

\numb{4} Remember that a whole number greater than 1 is \textit{prime} if its only proper divisor is 1. 

\bigskip

a) How many prime numbers are there? Does this help to answer 3a) (if you haven't already)?

\bigskip

b) If $p$ is a prime number, is $2^p$ a prime number? How about $2^p-1$?

\newpage

%Mersenne primes generate perfect numbers; part a) is rather hard, and part d) is open. 

\numb{5} A prime number of the form $2^p-1$ where $p$ is prime is called a \textit{Mersenne prime}.

\bigskip

a) Can you show that any number of the form $(2^p-1)(2^{p-1})$ where $2^p-1$ is a Mersenne prime is perfect?

\bigskip

b) Calculate the 3rd and 4th perfect numbers using a).

\bigskip

c) Can you show that any perfect number obtained by the rule in a) ends in a 6 or 8?

\bigskip

d) How many Mersenne primes do you think there are? Can you convince everyone else?

\newpage

%exactly what it says: strange facts about perfect numbers. This is really only for students who are really into number theory. 

\numb{6} Here are some bizarre facts about perfect numbers for you to think about. See if you can get at why they should be true. 

\bigskip

a) The only perfect number that is the sum of two cubes is 28.

\bigskip

b) The only perfect number that is not divisible by a perfect square is 6. 


\newpage

\numb{7} These last two problems are challenging.

%part a) is the converse to 5a). Part b) is open!

\bigskip

a) Show that any EVEN perfect number is of the form $(2^p-1)(2^{p-1})$ where $2^p-1$ is a Mersenne prime.

\bigskip

b) Find an odd perfect number. To get you going, here are some facts about such a number: it has at least 101 prime factors and at least 10 distinct prime factors, and if 3 is not one of the factors, then it has at least 12 distinct prime factors. 





\newpage

\begin{mdframed}[outerlinecolor=black,outerlinewidth=2pt,linecolor=cccolor,middlelinewidth=3pt,roundcorner=10pt]
  This work is licensed under a Creative Commons Attribution-NonCommercial 4.0 International License.
  \begin{center}
    \includegraphics[scale=2]{CCImage.png}
  \end{center}
\end{mdframed}



\bigskip


\end{document}

<a rel="license" href="http://creativecommons.org/licenses/by-nc/4.0/"><img alt="Creative Commons License" style="border-width:0" src="https://i.creativecommons.org/l/by-nc/4.0/88x31.png" /></a><br /><span xmlns:dct="http://purl.org/dc/terms/" property="dct:title">Perfect Numbers Math Circles Worksheet</span> by <span xmlns:cc="http://creativecommons.org/ns#" property="cc:attributionName">Alan Wiggins</span> is licensed under a <a rel="license" href="http://creativecommons.org/licenses/by-nc/4.0/">Creative Commons Attribution-NonCommercial 4.0 International License</a>.